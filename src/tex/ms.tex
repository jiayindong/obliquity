% Define document class
\documentclass[twocolumn,times]{aastex631}
\usepackage{showyourwork}
\usepackage{xspace}
\usepackage{amsmath}
\usepackage{amssymb}
\usepackage{bm}

\newcommand{\numistar}{65\xspace}
\newcommand{\numall}{161\xspace}

\received{}
\revised{}
\accepted{}

\submitjournal{the AAS Journals}

% Begin!
\begin{document}

% Title
\title{3D Stellar Obliquity Distribution of Exoplanetary Systems}

\correspondingauthor{Jiayin Dong}
\email{jdong@flatironinstitute.org}

\newcommand{\FlatironCCA}{Center for Computational Astrophysics, Flatiron Institute, 162 Fifth Avenue, New York, NY 10010, USA}

\author[0000-0002-3610-6953]{Jiayin Dong}
\altaffiliation{Flatiron Research Fellow}
\affiliation{\FlatironCCA}

\author[0000-0002-9328-5652]{Dan Foreman-Mackey}
\affiliation{\FlatironCCA}

% Abstract with filler text
\begin{abstract}
% We build a hierarchical Bayesian framework to constrain true stellar obliquity ($\psi$) distribution in exoplanetary systems.
% Using the sky-projected stellar obliquities ($\lambda$) of all measured systems, we infer their $\cos{\psi}$ distribution using a four-parameter Beta mixture model with the assumption of a uniform cosine of stellar inclination ($i_\star$) distribution for each planet. We find $\cos{\psi}$ has a significant pile-up near 1, suggesting the majority of exoplanetary systems are well aligned. For misaligned systems, $\cos{\psi}$ is nearly uniformly distributed between $-0.75$ to $0.75$ with no significant pipe-up near $\cos{\psi} = 0$, suggesting the stellar spin axis is randomly distributed around the planet's orbital axis.
% Our Bayesian framework could reproduce the excess of perpendicular planets observed in systems with constrained $i_\star$. However, we find the sample of misaligned systems could be biased because of the small sample size and that misaligned systems are often found around hot stars, whereas the gravity-darkening method, a technique commonly used to constrain the inclinations of hot stars ($T_{\rm eff} > 7000$K), is only sensitive to detect perpendicular planets.
We build a hierarchical Bayesian framework to infer the 3D stellar obliquity $\psi$ distribution in exoplanetary systems. 
First, we apply our framework to \numistar systems with sky-projected stellar obliquity $\lambda$ and stellar inclinations $i_\star$ measurements. The $\cos{\psi}$ distribution has a significant pile-up at 1, suggesting the majority of systems are well aligned, and has a second peak near $-0.15$ (i.e., $\psi \sim 100\degr$), consistent with the finding of a preponderance of perpendicular planets in previous work.
Interestingly, if we just use isotropic priors on $i_\star$ and no additional information, a similar $\cos{\psi}$ distribution is inferred, suggesting the $cos\psi$ distribution depends mainly on the underlying $\lambda$ distribution. 
We find the $\lambda$ distribution of misaligned systems in the measured-$i_\star$ sample clustered at $110 \degr$, whereas of all \numall systems with sky-projected obliquity measurements is nearly uniform. The measured-$i_\star$ sample may not fully represent the population. We discuss two possible reasons: (1) the measured-$i_\star$ sample only has a small number of misaligned systems, and (2) misaligned systems are often found around hot stars, and to measure the $i_\star$ of host stars ($T_{\rm eff} > 7000$K), the gravity-darkening method is commonly used but the method is biased towards detecting perpendicular planets.
Finally, we apply our Bayesian framework to all \numall systems and assume uninformative $i_\star$ priors. The $\cos{\psi}$ piles up at 1 but is nearly uniformly distributed between $-0.75$ to $0.75$, corresponding to $\psi$ of $140\degr$ and $40\degr$. We conclude that in misaligned exoplanetary systems, the stellar spin axis is isotropically distributed around its planets' orbital axis but lacks $\psi > 140\degr$.
\end{abstract}

% Main body with filler text
\section{Introduction}
\label{sec:intro}

3D stellar obliquity $\psi$ is described as the angle between a planet's orbital axis $\bf{i}_{\rm orb}$ and its host star's spin axis $\bf{i}_{\star}$.
For any given $\bf{i}_{\rm orb}$ axis, if $\bf{i}_{\star}$ is a random vector with uniform distribution on a three-dimensional sphere, the cosine of their separation angle $\cos{\psi}$ will be uniformly distributed between $-1$ and $1$.
In reality, we could only measure the sky-projected stellar obliquity $\lambda$, the angle between the projections of $\bf{i}_{\rm orb}$ and $\bf{i}_{\star}$ onto the plane of the sky. Stellar obliquity $\psi$ can be calculated from $\bf{i}_{\rm orb}$ and $\bf{i}_{\star}$ as the following:
\begin{equation}\label{eqn:psi}
    \cos{\psi} = \sin{i_\star}\sin{i_{\rm orb}}\cos{\lambda} + \cos{i_\star}\cos{i_{\rm orb}},
\end{equation}
where $i_{\rm orb}$ is the angle between the vector $\bf{i}_{\rm orb}$ and the observer's line of sight, $i_\star$ is the angle between $\bf{i}_{\star}$ and the observer's line of sight, and $\lambda$ is the sky-projected stellar obliquity. A full derivation of Equation~(\ref{eqn:psi}) can be found in, e.g., \cite{Fabrycky09} Equation (1)--(10).
Since the sky-projected stellar obliquity can only be measured in transiting exoplanet systems, $i_{\rm orb} \approx 90\degr$ and thus $\cos{\psi} \approx \sin{i_\star}\cos{\lambda}$.

To constrain the 3D stellar obliquity $\psi$, both a measurement of the sky-projected obliquity $\lambda$ and the stellar inclination $i_\star$ are required. The stellar inclination can often be constrained on exoplanetary systems via, e.g., photometric and spectroscopic rotational modulation introduced by starspots, gravity darkening, or asteroseismology. For systems without measured $i_\star$, it is still possible to infer the 3D stellar obliquity $\psi$ from the sky-projected obliquity $\lambda$ \citep[e.g.,][]{Fabrycky09}. To do so, we assume the stellar inclination $i_\star$ is isotropic, and thus the probability density function of ${i_\star} \sim \sin{i_{\star}}$, or $\cos{i_\star} \sim \mathcal{U}[0,1]$.

\cite{Albrecht21} inferred the 3D stellar obliquities of a sample of stars that have both stellar inclination and sky-projected stellar obliquity constraints. They found while the majority of planets have a small $\psi$, misaligned planets have $\psi$ clustered between 80--125$\degr$. Furthermore, they found the $\cos{\psi}$ distribution of misaligned systems is not uniformly distributed but clusters near 0, suggesting the stellar spin axis is not isotropically distributed around the planetary orbital axis.

In this work, we build a hierarchical Bayesian framework to evaluate the distribution of $\cos{\psi}$ using the sample in \cite{Albrecht21} and extend the analysis to the full sample of exoplanetary systems that have sky-projected stellar obliquity measurements extracted from the \cite{Albrecht22} review.

\section{Stellar Obliquity Distribution} \label{sec:hbm}
We build a hierarchical Bayesian framework to model the 3D stellar obliquity ($\psi$) distribution of exoplanetary systems.

First, we apply the framework to the sample of \numistar systems that have both sky-projected stellar obliquity $\lambda$ and stellar inclination $i_\star$ constraints.
In Figure~\ref{fig:graph}, we illustrate the probabilistic graphical model. We aim to constrain a set of hyperparameters $\bm{\beta}$ that describe the 3D stellar obliquity $\psi$ distribution. $\bm{\beta}$ is constrained by $N$ individual systems, where each system has $\psi_n$ that is calculated by the sky-projected stellar obliquity $\lambda_n$ and the stellar inclination $i_n$. For simplicity, we approximate $i_{\rm orb}$ to $90\degr$ in the model and use Equation~(\ref{eqn:psi}) to compute $\psi_n$. 
The parameter $\theta_{\star, n}$ contains information on stellar rotation period and stellar radius, if available.
If the rotational modulation method is used to constrain the stellar inclination \citep[e.g.,][]{Masuda20}, $i_{\star,n}$ and $\theta_{\star,n}$ are constrained by observed sky-projected stellar rotational line broadening $\hat{v}\sin{i_{\star,n}}$ and observed $\hat{\theta}_{\star,n}$. Lastly, the sky-projected stellar obliquity $\lambda_n$ is constrained by observed $\hat{\lambda}_n$.

\begin{figure}
    \script{graph.py}
    \includegraphics[width=\linewidth]{figures/graph.pdf}
    \caption{Hierarchical Bayesian framework to infer the stellar obliquity distribution of exoplanetary systems. The stellar obliquity distribution is described by hyperparameters $\beta$ and constrained by $n_{\rm th}$ system's stellar obliquity $\psi_n$. Each $\psi_n$ is calculated by the system's sky-projected stellar obliquity $\lambda_n$ and stellar inclination $i_{\star, n}$. $\theta_{\star, n}$ contains other information of the star other than $i_{\star, n}$, such as the stellar radius and stellar rotation period.}
    \label{fig:graph}
\end{figure}

For the stellar obliquity distribution, we adopt a two-component mixture Beta distribution with hyperparameters $\bm{\theta} = \{\bm{w},\bm{a},\bm{b}\}$, where each hyperparameter has a dimension of 2. The Beta distribution is flexible to describe both the spike of aligned systems with 3D stellar obliquities close to $0\degr$ and the clustered or broadly distributed systems with large obliquities.
The probability distribution follows 
\begin{align}
    \cos{\psi} &\sim 2\times\Bigl( w_0 {\rm Beta}(a_0, b_0) + w_1 {\rm Beta}(a_1, b_1)\Bigr)-1.
\end{align}
Since a Beta distribution is defined on the interval $[0, 1]$, we extend its support to $[-1,1]$ using a linear transformation (i.e., $2\times Beta-1$).

The hyperparameter and parameter priors are the following:
\begin{align}
    w_{0,1} &\sim {\rm Dirichlet}(1,1) \nonumber\\
    a_0 &\sim \mathcal{U}(0,50) \nonumber\\
    b_0 &\sim \mathcal{U}(0,1) \nonumber\\
    a_1 &\sim \mathcal{U}(0,10) \nonumber\\
    b_1 &\sim \mathcal{U}(0,10) \nonumber\\
    \cos{i}_{\star,n} &\sim \mathcal{U}(0,1) \nonumber\\
    \lambda_{\star,n} &\sim \mathcal{U}(0,\pi).
\end{align}
We set the hyperpriors such that $a_0$ and $b_0$ will be used to infer the spike of $\cos{\psi}$ near 1, and $a_1$ and $b_1$ infer the distribution of the misaligned population. Doing so will also avoid label switching in our mixture model.

The likelihood functions are the following:
\begin{align}
    L(\lambda) &\sim \prod_{i=1}^N\mathcal{N}(\hat{\lambda}_n, \sigma_{\hat{\lambda}_n}) \nonumber\\
    L(i_\star, \theta_\star) &\sim \prod_{i=1}^N\mathcal{N}(\hat{v}\sin{i_{\star,n}}, \sigma_{\hat{v}\sin{i_{\star,n}}}) \nonumber\\
    L(\theta) &\sim \prod_{i=1}^N\mathcal{N}(\hat{\theta}_n, \sigma_{\hat{\theta}_n}).
\end{align}

\begin{figure*}[ht!]
    \script{psi_dist.py}
    \begin{centering}
        \includegraphics{figures/psi_dist.pdf}
        \caption{The cosine of 3D stellar obliquity $\cos{\psi}$ distributions.}
        \label{fig:psi_dist}
    \end{centering}
\end{figure*}

We apply the model to \numistar systems with both constrained $\lambda$ and $i_\star$ summarized in \cite{Albrecht21}. The inferred $\cos{\psi}$ distribution is shown in the left panel of Figure~\ref{fig:psi_dist}. The distribution presents a pike at 1 as expected, suggesting the majority of exoplanetary systems are well-aligned systems. For misaligned systems, $\cos{\psi}$ presents a cluster near $\cos{\psi} = -0.2$, corresponding to $\psi$ of $100\degr$. The result is consistent with the findings of a preponderance of perpendicular planets in \cite{Albrecht21}.

To test the dependence of $\cos{\psi}$ distribution on $i_\star$ measurements, we apply the model again to the sample of \numistar systems with both constrained $\lambda$ and $i_\star$. However, in this model, we do not use any additional information on $i_\star$ other than applying an isotropic inclination prior to each system. The corresponding probabilistic graphical model is Figure~\ref{fig:graph} but removing all the dashed edges. 
The $\cos{\psi}$ distribution of the new model is shown in the middle panel of Figure~\ref{fig:psi_dist}. Interestingly, the overall $\cos{\psi}$ distributions of the two models are very similar, despite that the inferred $\psi$ uncertainty for each system is now greater.
The similarity between the two distributions suggests that, at least for the measured-$i_\star$ sample, the $\cos{\psi}$ distribution mainly depends on the underlying sky-project stellar obliquity $\lambda$ distribution.

Following the logic, we now apply our model to all \numall exoplanetary systems that have sky-projected stellar obliquity measurements summarized in \cite{Albrecht22}. The result is presented in the right panel of Figure~\ref{fig:psi_dist}. The $\cos{\psi}$ distribution still spikes at 1 but is nearly flat between $-0.75$ and $0.75$ with no significant clustering. The distribution corresponds to an isotropic distribution of $\psi$ from $40\degr$ to $140\degr$.

We also test the robustness of $\cos{\psi}$ distribution on different hyperprior choices \citep[see a nice summary in][]{Nagpal22}. We find extending the support of hyperpriors from 0--10 to 0--100 or 0--1000 or changing hyperprior distributions from Uniform to log-Uniform or truncated-Normal all infer similar results to Figure~\ref{fig:psi_dist}.

Although the distribution of $\psi$ looks very similar for the $i_\star$ sample using constrained or random priors, the uncertainty on each $\psi$ is much tighter. 

\section{Hypothesis} \label{sec:tests}

\begin{figure*}[ht!]
    \script{lam_dist.py}
    \begin{centering}
        \includegraphics{figures/lam_dist.pdf}
        \caption{Sky-projected stellar obliquity ($\lambda$) distributions of \numistar systems with constrained $i_\star$ and all \numall exoplanetary systems that have measured sky-projected stellar obliquities.}
        \label{fig:lam_dist}
    \end{centering}
\end{figure*}


\subsection{Stellar Teff $>$ 7000K?}
All planets with host star effective temperatures greater than 7000 K are near polar.
Note that all these stars are expected to be fast rotators that limit the Rossiter-McLaughlin effect measurement. Doppler Tomography can be applied to these stars, but the majority of the measurements are conducted by gravitational darkening that is biased toward polar planets.

We removed 7/21 planets with host star effective temperatures greater than 7000 K and redo the analysis. Interestingly, we find the results do not change. See Figure~\ref{fig:teff_cutff}.

% \begin{figure*}[ht!]
%     \script{teff_cut.py}
%     \begin{centering}
%         \includegraphics[width=\linewidth]{figures/teff_cut.pdf}
%         \caption{$\cos{\psi}$ distributions for the $i_\star$ sample including/not including $>$7000 K stars.}
%         \label{fig:teff_cut}
%     \end{centering}
% \end{figure*}

\subsection{Small vsini}
Albrecht et al. 21 that underestimating $v\sin{i_\star}$, overestimating $R_\star$, or underestimating $P_{\rm rot}$ could all lead to biased estimation on $\cos{\psi}$. We performed another test to evaluate the sample with random $i_\star$ prior (i.e., uniform on $\cos{i}$). Interestingly, we find a very similar distribution to the $\cos{\psi}$ distributions inferred with informative $i_\star$ priors. See Figure~\ref{fig:rand_inc}.

% \begin{figure*}[ht!]
%     \script{rand_inc.py}
%     \begin{centering}
%         \includegraphics[width=\linewidth]{figures/rand_inc.pdf}
%         \caption{$\cos{\psi}$ distributions for the $i_\star$ sample with/without informative $i_\star$ priors.}
%         \label{fig:rand_inc}
%     \end{centering}
% \end{figure*}

\subsection{Statistical fluke}
Could it just be a statistical fluke? We perform the test by randomly drawing 20 planets out of the 40 misaligned systems. We find in half of the time, we could not reproduce the distribution but result in a pile-up of $\cos{\psi}$. See Figure~\ref{fig:bootstrap}.

% \begin{figure*}[ht!]
%     \script{bootstrap.py}
%     \begin{centering}
%         \includegraphics[width=\linewidth]{figures/bootstrap.pdf}
%         \caption{$\cos{\psi}$ distributions for the $i_\star$ sample with/without informative $i_\star$ priors.}
%         \label{fig:bootstrap}
%     \end{centering}
% \end{figure*}

\section{Discussion}
No strong clustering?

\bibliography{bib}

\end{document}
