% Define document class
\documentclass[twocolumn]{aastex631}
\usepackage{showyourwork}

\received{}
\revised{}
\accepted{}

\submitjournal{the AAS Journals}

% Begin!
\begin{document}

% Title
\title{True Stellar Obliquity Distribution of Exoplanets}

\correspondingauthor{Jiayin Dong}
\email{jdong@flatironinstitute.org}

\newcommand{\FlatironCCA}{Center for Computational Astrophysics, Flatiron Institute, 162 Fifth Avenue, New York, NY 10010, USA}

\author[0000-0002-3610-6953]{Jiayin Dong}
\altaffiliation{Flatiron Research Fellow}
\affiliation{\FlatironCCA}

\author[0000-0002-9328-5652]{Dan Foreman-Mackey}
\affiliation{\FlatironCCA}

% Abstract with filler text
\begin{abstract}
We build a hierarchical Bayesian framework to constrain true stellar obliquity ($\psi$) distribution in exoplanetary systems.
Using the sky-projected stellar obliquities ($\lambda$) of a sample of 151 exoplanets, we infer their $\cos{\psi}$ distribution using a four-parameter Beta mixture model with the assumption of a uniform cosine of stellar inclination ($i_\star$) distribution for each planet. We find $\cos{\psi}$ has a significant pile-up near 1, suggesting the majority of exoplanetary systems are well aligned. For misaligned systems, $\cos{\psi}$ is nearly uniformly distributed between $-0.75$ to $0.75$ with no significant pipe-up near $\cos{\psi} = 0$, suggesting the stellar spin axis is randomly distributed around the planet's orbital axis.
Our Bayesian framework could reproduce the excess of perpendicular planets observed in systems with constrained $i_\star$. However, we find the sample of misaligned systems could be biased because of the small sample size and that misaligned systems are often found around hot stars, whereas the gravity-darkening method, a technique commonly used to constrain the inclinations of hot stars ($T_{\rm eff} > 7000$K), is only sensitive to detect perpendicular planets. 
\end{abstract}

We build a hierarchical Bayesian framework to constrain true stellar obliquity ($\psi$) distribution in exoplanetary systems.

% Main body with filler text
\section{Introduction}
\label{sec:intro}

Stellar obliquity $\psi$ is described as the angle between a planet's orbital axis $\bf{i}_{\rm orb}$ and its host star's spin axis $\bf{i}_{\star}$.
If both $\bf{i}_{\rm orb}$ and $\bf{i}_{\star}$ are independent, random unit vectors with uniform distribution on a three-dimensional sphere, $\cos{\psi}$ will be uniformly distributed between $-1$ and $1$.
In reality, the sky-projected stellar obliquity $\lambda$ can only be measured in transiting exoplanet systems, limiting $\bf{i}_{\rm orb}$ close to $\sim 90\degr$.

Stellar obliquity $\psi$ can be calculated from $\bf{i}_{\rm orb}$ and $\bf{i}_{\star}$ as the following:
\begin{equation}
    \cos{\psi} = \sin{i_\star}\sin{i_{\rm orb}}\cos{\lambda} + \cos{i_\star}\cos{i_{\rm orb}},
\end{equation}
where $i_{\rm orb}$ and $i_\star$ are the projections of $\bf{i}_{\rm orb}$ and $\bf{i}_{\star}$ along the observer's line of sight, and $\lambda$ is the sky-projected stellar obliquity. A full derivation can be found in, e.g., \cite{Fabrycky09} Equation (1)--(10).
When setting $i_{\rm orb} = 90\degr$, $\cos{\psi} = \sin{i_\star}\cos{\lambda}$.

In most exoplanetary systems, only the sky-projected obliquity $\lambda$ is constrained. We could still infer $\psi$ given the measured $\lambda$ and assuming a randomly distributed $i_\star$ that is uniformly distributed in $\cos{i_\star}$.
In some systems, $i_\star$ can be constrained via photometric rotational modulation, gravity darkening, or asteroseismology. In such systems, an informative $i_\star$ prior can be used.

\cite{Albrecht21} inferred the true stellar obliquities of a sample of stars that have both stellar inclinations and sky-projected stellar obliquity constraints. They found while the majority of planets have a small $\psi$, misaligned planets have $\psi$ clustered between 80--125$\degr$. Furthermore, they found the $\cos{\psi}$ distribution of misaligned systems is not uniformly distributed but clustered near 0, suggesting the stellar spin axis is not randomly distributed around the planetary orbital axis.

In this work, we build a hierarchical Bayesian framework to evaluate the distribution of $\cos{\psi}$ using the sample in \cite{Albrecht21} and extend the analysis to the full sample of exoplanetary systems that have sky-projected stellar obliquity measurements extracted from the review paper of \cite{Albrecht22}.

\section{Stellar Obliquity Distribution} \label{sec:hbm}
We model the whole population stellar obliquity distribution and also the sub-population with $i_\star$ constrained. See Figure\ref{fig:hbm} as an illustration. 

\begin{figure}[ht!]
    \script{hbm.py}
    \includegraphics[width=\linewidth]{figures/hbm.pdf}
    \caption{Hierarchical Bayesian framework to infer stellar obliquity distribution described by hyperparameters $\theta$ from individual system's stellar obliquity $\psi_n$. $\psi_n$ is calculated by the system's sky-projected stellar obliquity $\lambda_n$ and stellar inclination $i_{\star, n}$.}
    \label{fig:hbm}
\end{figure}

We use a four-parameter Beta distribution. The framework is shown in Figure~\ref{fig:hbm}.

\begin{itemize}
    \item Uniform $\cos{i}$
    \item Uniform $\lambda$
    \item Observed Normal $\lambda$
    \item Observed Normal $R_\star$
    \item Observed Normal $P_{\rm rot}$
\end{itemize}

We find the $i_\star$ sample, there is a pile-up of planets near $-0.25$ whereas, for the all-planet sample, the distribution is mostly flat. See Figure~\ref{fig:psi_dist}.

What makes the difference?

\cite{Nagpal22} emphasize the importance of picking the right hyperpriors.

\begin{figure*}[ht!]
    \script{psi_dist.py}
    \begin{centering}
        \includegraphics[width=\linewidth]{figures/psi_dist.pdf}
        \caption{$\cos{\psi}$ distributions.}
        \label{fig:psi_dist}
    \end{centering}
\end{figure*}

\section{Hypothesis} \label{sec:tests}

\subsection{Stellar Teff $>$ 7000K?}
All planets with host star effective temperatures greater than 7000 K are near polar.
Note that all these stars are expected to be fast rotators that limit the Rossiter-McLaughlin effect measurement. Doppler Tomography can be applied to these stars, but the majority of the measurements are conducted by gravitational darkening that is biased toward polar planets.

We removed 7/21 planets with host star effective temperatures greater than 7000 K and redo the analysis. Interestingly, we find the results do not change. See Figure~\ref{fig:teteff_cutff}.

% \begin{figure*}[ht!]
%     \script{teff_cut.py}
%     \begin{centering}
%         \includegraphics[width=\linewidth]{figures/teff_cut.pdf}
%         \caption{$\cos{\psi}$ distributions for the $i_\star$ sample including/not including $>$7000 K stars.}
%         \label{fig:teff_cut}
%     \end{centering}
% \end{figure*}

\subsection{Small vsini}
Albrecht et al. 21 that underestimating $v\sin{i_\star}$, overestimating $R_\star$, or underestimating $P_{\rm rot}$ could all lead to biased estimation on $\cos{\psi}$. We performed another test to evaluate the sample with random $i_\star$ prior (i.e., uniform on $\cos{i}$). Interestingly, we find a very similar distribution to the $\cos{\psi}$ distributions inferred with informative $i_\star$ priors. See Figure~\ref{fig:rand_inc}.

% \begin{figure*}[ht!]
%     \script{rand_inc.py}
%     \begin{centering}
%         \includegraphics[width=\linewidth]{figures/rand_inc.pdf}
%         \caption{$\cos{\psi}$ distributions for the $i_\star$ sample with/without informative $i_\star$ priors.}
%         \label{fig:rand_inc}
%     \end{centering}
% \end{figure*}

\subsection{Statistical fluke}
Could it just be a statistical fluke? We perform the test by randomly drawing 20 planets out of the 40 misaligned systems. We find in half of the time, we could not reproduce the distribution but result in a pile-up of $\cos{\psi}$. See Figure~\ref{fig:bootstrap}.

% \begin{figure*}[ht!]
%     \script{bootstrap.py}
%     \begin{centering}
%         \includegraphics[width=\linewidth]{figures/bootstrap.pdf}
%         \caption{$\cos{\psi}$ distributions for the $i_\star$ sample with/without informative $i_\star$ priors.}
%         \label{fig:bootstrap}
%     \end{centering}
% \end{figure*}

\section{Discussion}
No strong clustering?

\bibliography{bib}

\end{document}
