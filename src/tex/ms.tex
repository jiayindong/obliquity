% Define document class
\documentclass[twocolumn,times]{aastex631}
\usepackage{showyourwork}
\usepackage{xspace}
\usepackage{amsmath}
\usepackage{amssymb}
\usepackage{bm}
\usepackage{physics}
\usepackage[super]{nth}
\usepackage{fontawesome}
\usepackage{hyperref}

\received{}
\revised{}
\accepted{}

\submitjournal{the AAS Journals}

% Begin!
\begin{document}

% Title
\title{A Hierarchical Bayesian Framework for Exoplanetary Stellar Obliquity Distribution}

\correspondingauthor{Jiayin Dong}
\email{jdong@flatironinstitute.org}

\newcommand{\FlatironCCA}{Center for Computational Astrophysics, Flatiron Institute, 162 Fifth Avenue, New York, NY 10010, USA}

\author[0000-0002-3610-6953]{Jiayin Dong}
\altaffiliation{Flatiron Research Fellow}
\affiliation{\FlatironCCA}

\author[0000-0002-9328-5652]{Dan Foreman-Mackey}
\affiliation{\FlatironCCA}

\author{et al.}

\received{}
\revised{}
\accepted{}

\submitjournal{the AAS Journals}

% Abstract with filler text
\begin{abstract}

Stellar obliquity $\psi$, the angle between a planet's orbital axis and its host star's spin axis, traces the formation and evolution of a planetary system. In transiting-exoplanet observations, only the sky-projected stellar obliquity $\lambda$ can be measured. To find the true stellar obliquity $\psi$, information about stellar inclination $i_\star$ is also required.
Here we show that, while the stellar inclination is important to infer the stellar obliquity of an individual system, they are not necessarily required for the population stellar obliquity inference. 
Constraints on the stellar obliquity distribution are predominantly contributed by the sky-projected stellar obliquities.
We introduce a flexible, hierarchical Bayesian framework that could infer the stellar obliquity distribution purely from sky-projected stellar obliquities. Stellar inclination measurements are optional in the model, and if not provided, are assumed to be isotropically distributed.
Applying the framework to all exoplanetary systems with sky-projected stellar obliquity measurements, we find that the stellar obliquity distribution is unimodal and peaked at zero degrees. Misaligned systems have nearly isotropic stellar obliquities, which may have important implications for the formation and evolution of planetary systems.

\end{abstract}

% Main body with filler text
\section{Introduction}
\label{sec:intro}

Stellar obliquity $\psi$ describes the angle between a planet's orbital axis $\bf{n}_{\rm orb}$ and its host star's spin axis $\bf{n}_{\star}$. 
The angle is an important tracer of a planetary system's formation environment and dynamical evolution. The evolution of stellar obliquity can be roughly summarized into three stages. First, the formation and evolution of a protoplanetary disk around its host star determine the primordial stellar obliquity \citep[e.g.,][]{Bate10, Lai11, Batygin12}. Second, post-formation dynamical evolution in the planetary system, such as planet-planet scattering \citep[e.g.,][]{Rasio96, Chatterjee08, Nagasawa08, Beague12}, von Zeipel-Kozai-Lidov mechanisms \citep[e.g.,][]{Wu03, Naoz16}, and secular chaos \citep{Wu11}, excites the mutual inclinations between planetary or stellar companions, further sculpting the stellar obliquity. Lastly, tidal realignment of the host star's spin axis to the planet's orbital axis could reduce the stellar obliquity, if the tidal dissipation in the star is efficient \citep[e.g.,][]{Winn10, Albrecht12}. Furthermore, massive stars with convective cores could generate internal gravity waves and dissipate angular momentum to their radiative zones, modifying the stellar obliquity \citep{Rogers12, Rogers13}.

It is yet unclear if all of these physical and dynamic processes apply to exoplanetary systems. These proposed mechanisms have different predictions on stellar obliquity distributions \citep[see][and references therein]{Albrecht22}. For example, in the secular chaos mechanism, the stellar obliquity could never get beyond $90\degr$ \citep{Teyssandier19}. In the stellar Kozai mechanism, the stellar obliquity distribution should be bimodal and concentrated at $40\degr$ and $140\degr$ \citep[e.g.,][]{Fabrycky07, Anderson16, Vick19}. In the multiple-planet scattering mechanism, the majority of systems should be aligned, and a small fraction of systems have a wide range of stellar obliquities \citep[e.g.,][]{Beague12}. Motivated by these predictions, we aim to constrain the stellar obliquity distribution of exoplanetary systems from a Bayesian approach to find dominant mechanisms that sculpt close-in planetary systems.

In exoplanetary observation, only the sky-projected stellar obliquity $\lambda$, the angle between the projections of $\bf{n}_{\rm orb}$ and $\bf{n}_{\star}$ onto the plane of the sky, can be measured. Stellar obliquity $\psi$ can be calculated, if the stellar inclination $i_\star$ is known. The relation follows \citep[e.g.,][]{Fabrycky09}:
\begin{equation}\label{eqn:psi}
    \cos{\psi} = \sin{i_\star}\sin{i_{\rm orb}}\cos{\lambda} + \cos{i_\star}\cos{i_{\rm orb}},
\end{equation}
where $i_{\rm orb}$ is the angle between the vector $\bf{n}_{\rm orb}$ and the observer's line of sight, and $i_\star$ is the angle between $\bf{n}_{\star}$ and the observer's line of sight.
For transiting exoplanet systems where $i_{\rm orb} \approx 90\degr$, $\cos{\psi} \approx \sin{i_\star}\cos{\lambda}$.

Stellar inclination can often be constrained via, e.g., photometric and spectroscopic rotational modulation introduced by starspots, gravity darkening, and asteroseismology \citep[see][and references therein]{Albrecht22}.
For systems without $i_\star$ measurements, it is still possible to infer their stellar obliquities from the sky-projected obliquities, assuming isotropic stellar inclinations. The inferred $\psi$ will have greater uncertainty than the one inferred with $i_\star$ measurement \citep{Fabrycky09}.

We do not yet understand the importance of stellar inclination in population stellar obliquity inference.
In this work, we infer the stellar obliquity distribution from sky-projected stellar obliquities and show that the stellar obliquity distribution weakly depends on stellar inclinations.
In Section~\ref{sec:hbm}, we introduce a flexible, hierarchical Bayesian framework to infer the population stellar obliquity distribution.
In Section~\ref{sec:applications}, we apply the framework to simulated data and real observations.
In Section~\ref{sec:jacobian}, we find the mathematical relation between the $\psi$, $\lambda$, and $i_\star$ distributions and explain why stellar inclination is not important in stellar obliquity distribution inference.

\section{Hierarchical Bayesian Framework}\label{sec:hbm}

To find the stellar obliquity distribution of exoplanetary systems, we build a hierarchical Bayesian framework that takes the sky-projected stellar obliquity $\lambda$ and orbital inclination $i_{\rm orb}$ as observed data. Measurements on the stellar inclination $i_\star$ are optional. If the $i_\star$ information is not provided, we simply assume that they follow an isotropic distribution. Otherwise, the observed $i_\star$ could be taken from previous measurements, or if the rotational modulation method is applied, the $i_\star$ could be inferred from the stellar rotation period $P_{\rm rot}$, stellar radius $R_\star$, and sky-projected rotational broadening velocity $v\sin{i}_\star$.

\begin{figure}[ht!]
    \script{graph.py}
    \includegraphics[width=\linewidth]{figures/graph.pdf}
    \caption{Hierarchical Bayesian framework to infer the stellar obliquity distribution of exoplanetary systems. The stellar obliquity distribution is described by hyperparameters $\vb*{\beta}$ and constrained by $n_{\rm th}$ system's stellar obliquity $\psi_n$. Each $\psi_n$ is calculated by the system's sky-projected stellar obliquity $\lambda_n$, orbital inclination $i_{{\rm orb}, n}$, and stellar inclination $i_{\star, n}$, if available. $\gamma_{\star, n}$ contains properties of the star other than inclination, such as stellar radius and rotation period for $i_{\star, n}$ inference. Obs$_{\star,n}$ includes all the observed properties of the star with uncertainties, $\hat{\lambda}_n$ has the measured sky-projected stellar obliquity and its uncertainty, and $\hat{i}_{{\rm orb}, n}$ has measured orbital inclination and its uncertainty.}
    \label{fig:graph}
\end{figure}

Figure~\ref{fig:graph} shows the probabilistic graphical model for the hierarchical Bayesian framework. We aim to constrain a set of hyperparameters $\bm{\beta}$ that describes the stellar obliquity $\psi$ distribution. The parameter set $\bm{\beta}$ is constrained by $N$ individual systems in which $\psi_n$ is a random variable. $\psi_n$ is computed from the sky-projected stellar obliquity $\lambda_n$, the orbital inclination $i_{{\rm orb},n}$, and the stellar inclination $i_{\star, n}$. 
The parameter $\gamma_{\star, n}$ is a parameter set that contains all the stellar properties other than $i_{\star,n}$, such as the stellar rotation period $P_{{\rm rot},n}$, radius $R_{\star,n}$, and projected rotational velocity $v\sin{i_{\star,n}}$, if they are available.

The sky-projected stellar obliquity $\lambda_n$ is constrained by observed $\hat{\lambda}_n$ via the Rossiter-McLaughlin effect or gravity darkening.
The orbital inclination $i_{{\rm orb},n}$ is constrained by the transit light curves of exoplanets.
Obs$_{\star,n}$ includes observed data that helps to constrain $i_{\star,n}$ and $\gamma_{\star, n}$.
If the rotational modulation method is used to constrain the stellar inclination, the $i_{\star,n}$ and $\gamma_{\star,n}$ are constrained by rotational modulation in photometric or spectroscopic time series and observed sky-projected stellar rotational line broadening $\hat{v}\sin{i_{\star,n}}$ \citep[e.g.,][]{Masuda20}, where we have $v\sin{i_\star} = 2 \pi R_\star / P_{\rm rot}$.
If the gravity-darkening method is used, the $i_{\star,n}$ could be constrained by the anomaly in transit light curves. If asteroseismology is used, the $i_{\star,n}$ could be constrained from the mode splitting in the Fourier transform of photometric time series.

For the stellar obliquity distribution, we model the $\cos{\psi}$ distribution instead of $\psi$ distribution to understand if the stellar obliquity is isotropically distributed. If stellar obliquity distribution follows an isotropic distribution, the $\cos{\psi}$ distribution will be uniformly distributed between -1 to 1. We adopt a two-component mixture Beta distribution with hyperparameters $\bm{\beta} = \{\bm{w},\bm{a},\bm{b}\}$, where each hyperparameter has a dimension of 2. One component is designed for a large population of well-aligned systems, and the other is designed for misaligned systems. The Beta distribution is a flexible distribution that can describe the spike of aligned systems with stellar obliquities close to $0\degr$ or the clustered or broadly distributed misaligned systems.
The probability density function of $\cos{\psi}$ follows 
\begin{align}
    \cos{\psi} &\sim 2\times\Bigl( w_0 {\rm Beta}(a_0, b_0) + w_1 {\rm Beta}(a_1, b_1)\Bigr)-1.
\end{align}
Since the Beta distribution is defined on the interval $[0, 1]$, we extend its support to $[-1,1]$ using a linear transformation (i.e., $2\times {\rm Beta} - 1$). For the ${\rm Beta}(a_0,b_0)$ component, we have it to describe the population of aligned systems. For the $a_0/b_0 \gg 1$, the distribution will have a spike at $\cos{\psi} = 1$. For the ${\rm Beta}(a_1,b_1)$ component, we have it to describe the misaligned systems. We design the order of the two components to avoid label switching in the mixture model.
With the design, the hyperparameter and parameter priors are the following:
\begin{align}
    w_{0,1} &\sim {\rm Dirichlet}(1, 1) \nonumber\\
    a_0 &\sim \mathcal{U}(0, 50) \nonumber\\
    b_0 &\sim \mathcal{U}(0, 1) \nonumber\\
    a_1 &\sim \mathcal{U}(0, 10) \nonumber\\
    b_1 &\sim \mathcal{U}(0, 10) \nonumber\\
    \cos{i}_{\star,n} &\sim \mathcal{U}(0, 1) \nonumber\\
    \cos{i}_{{\rm orb},n} &\sim \mathcal{U}(-1, 1) \nonumber\\
    \lambda_{\star,n} &\sim \mathcal{U}(0, \pi).
\end{align}
The likelihood functions follow:
\begin{align}
    \mathcal{L}(\lambda) &\sim \prod_{i=1}^N\mathcal{N}(\hat{\lambda}_n, \sigma_{\hat{\lambda}_n}) \nonumber\\
    \mathcal{L}(i_{\rm orb}) &\sim \prod_{i=1}^N\mathcal{N}(\hat{i}_{{\rm orb},n}, \sigma_{\hat{i}_{{\rm orb},n}}) \nonumber\\
    \mathcal{L}(i_\star) &\sim \prod_{i=1}^N\mathcal{N}({\rm Obs}_{\star,n}, \sigma_{{\rm Obs}_{\star,n}}).
\end{align}
If $\gamma_{\star, n}$ is available, we construct Normal distributions with means and standard deviations from their observed data.
Here we assume uniform hyperpriors for the Beta distribution. For population inference on a small number of systems (i.e., $N \lesssim 50$), the choice of hyperpriors could impact the inferred distribution \citep{Nagpal22}. When applying the framework to a small sample size, it is important to test the robustness of inferred distributions on different hyperpriors.

The probabilistic model is built with the $\mathtt{PyMC}$ \citep[$\mathtt{v4.1.7}$;][]{pymc} package and the posteriors are sampled using the No-U-Turn Sampler \citep[NUTS;][]{Hoffman11}, a gradient-based sampling algorithm of the Markov chain Monte Carlo (MCMC). All figures and simulations in this paper are fully reproducible and built with the $\mathtt{showyourwork}$ package. The open source code can be found on \href{https://github.com/jiayindong/obliquity}{GitHub\,\faGithub\,(https://github.com/jiayindong/obliquity)}.

\section{Applications to Simulated and Observational Data}\label{sec:applications}

\subsection{Simulated Data}\label{subsec:sims}

We first apply the hierarchical Bayesian framework to simulated data of which the ground-truth stellar obliquity distribution is known. We test the four $\cos{\psi}$ distributions: a uniform distribution $\mathcal{U}(-1,1)$ and three Normal distributions $\mathcal{N}(0,0.2)$, $\mathcal{N}(-0.4,0.2)$, and $\mathcal{N}(0.4,0.2)$.
For each $\cos{\psi}$ distribution, we randomly generate 200 samples of stellar inclination $i_\star$ and sky-projected stellar obliquity $\lambda$. We assume the stellar spin axis is uniformly distributed around the planetary orbital axis in the azimuthal direction and the orbital inclination is $90\degr$. 
The sampled $i_\star$ and $\lambda$ here are \emph{true} values. We then add some Gaussian noises to the \emph{true} $\lambda$ and $i_\star$ to simulate the \emph{observed} data. We choose uncertainties of $\sigma_{\lambda} = 8\degr$ and $\sigma_{i_\star} = 10\degr$, which are typical observational uncertainties summarized in \cite{Albrecht22}. Using the observed $\lambda$ and their uncertainties, we infer the $\cos{\psi}$ distribution of the sample with or without $i_\star$ likelihoods. 

In Figure~\ref{fig:simulation}, we present the inferred stellar obliquity distributions. Since the simulated stellar obliquity distributions only have a single component, we model the data with a single Beta distribution. Each row of Figure~\ref{fig:simulation} corresponds to an injected stellar obliquity distribution. The orange curve and contours are the median and 1-$\sigma$ and 2-$\sigma$ uncertainties of the inferred $\cos{\psi}$ distribution with stellar inclination information, and the blue curve and contours are the ones without stellar inclination information. We simply assume an isotropic distribution of $i_\star$ for these models.
Surprisingly, in all four distributions, the inferred $\cos{\psi}$ distributions with or without $i_\star$ correctly recover the injected distributions shown as grey dashed lines as shown in Figure~\ref{fig:simulation}. The inferred modes and widths of the stellar obliquity distributions with or without $i_\star$ likelihood are consistent, despite that the inferred distributions without $i_\star$ measurements could have wider uncertainties shown in shallow color contours.
Since the injected distributions are Normal distributions in rows 2--4, it is expected that the inferred distributions, which as Beta distributions, could not exactly describe the injected distributions.

We also examine the importance of orbital inclination $i_{\rm orb}$ to the stellar obliquity distribution inference. Since all the systems we study are transiting-planet systems, we consider an isotropic orbital inclination distribution between $80\degr$ and $90\degr$. This broad inclination range corresponds to an impact parameter range from 0 to 1 with a planet-star separation $a/R_\star$ of 6.
We compare the inferred stellar obliquity distribution by approximating $i_{\rm orb}$ to $90\degr$ with the inferred distribution using the true $i_{\rm orb}$, and find the difference between the two distributions is minimal. For transiting-planet systems, approximating orbital inclinations to $90\degr$ will not compromise the stellar obliquity distribution inference.

From the simulated data, we confirm that while stellar inclinations help to constrain the stellar obliquity of individual systems, they are not necessarily required for the population stellar obliquity inference. The constraint on stellar obliquity distribution is predominantly based on sky-projected stellar obliquities. Just from the sky-projected stellar obliquity, we could correctly infer the peak and width of the underlying stellar obliquity distribution. Later in Section~\ref{sec:jacobian}, we derive the Jacobian transformations between $\psi$, $\lambda$, and $i_\star$ to explain why there is no significant information loss when not having $i_\star$ measurements.

\begin{figure}[ht!]
    \script{sim.py}
    \includegraphics[width=\linewidth]{figures/simulation.pdf}
    \caption{Inferred stellar obliquity distributions from sky-projected stellar obliquities with or without stellar inclination measurements. Each row presents a set of simulated data where the underlying $\cos{\psi}$ distribution is known and shown as grey dashed curves. The inferred $\cos{\psi}$ distributions with or without stellar inclination measurements are shown in orange and blue curves, respectively. The shallow contours are 1-$\sigma$ and 2-$\sigma$ uncertainties of the inferred distributions.}
    \label{fig:simulation}
\end{figure}

\subsection{Exoplanetary Stellar Obliquity Distribution}

We next apply the hierarchical Bayesian framework to all exoplanetary systems with sky-projected stellar obliquity $\lambda$ measurements. We use the sample of 161 systems summarized in \cite{Albrecht22} Table A1, which are mainly Hot Jupiter systems. The inferred $\cos{\psi}$ distribution is shown in Figure~\ref{fig:psi_dist}. The $\cos{\psi}$ distribution peaks at 1 but is nearly flat between $-0.75$ and $0.75$ with no significant clustering. The distribution corresponds to a pile-up of planets with stellar obliquity less than $40\degr$ and an isotropic distribution of planets with stellar obliquity from $40\degr$ to $140\degr$. 
The aligned-system population dominates the distribution with a fraction of $w_0 = 0.67 \pm 0.09$. The hyperposteriors of the Beta distribution are $a_0 = 31.0\pm12.8$ and $b_0 = 0.40\pm0.12$.
The misaligned-system population has a fraction of $w_1 = 0.33 \pm 0.09$ and hyperposteriors of $a_1 = 1.56\pm0.93$ and $b_1 = 1.64\pm1.58$.

\begin{figure}[ht!]
    \script{psi_plot.py}
    \begin{centering}
        \includegraphics{figures/psi_dist.pdf}
        \caption{Stellar obliquity distribution of all exoplanetary systems with sky-projected stellar obliquity measurements. The inference is purely from observed sky-projected stellar obliquities and assumes the stellar inclination is isotropic.}
        \label{fig:psi_dist}
    \end{centering}
\end{figure}

The inferred stellar obliquity distribution of the full sample of systems is at odds with the previous analysis of the subsample of systems that also have $i_\star$ measurements. The subsample identified a preponderance of perpendicular planets that disfavored an isotropic stellar obliquity distribution \citep{Albrecht21}. We speculate two possible explanations for this observation discrepancy, which are open to further investigation.
First, the sample of systems with $i_\star$ measurements is small, and only about 20 systems are misaligned systems used to constrain the misaligned systems' stellar obliquity distribution. Second, the sample could have a selection bias introduced by the $i_\star$ measurement requirement.

The stellar obliquity distribution suggests that $67\pm9$\% of the systems have stellar obliquity less than $40\degr$, and $33\pm9$\% of the systems have nearly isotropically distributed stellar obliquities spread from $40\degr$ to $140\degr$. The distribution could have important implications for the formation and evolution of close-in planetary systems.
The broad distribution of misaligned systems is in good agreement with the outcome of the scattering of multiple giant planets after a convergent disk migration \citep[e.g.,][]{Nagasawa11, Beague12}, one of the proposed mechanisms that form Hot Jupiters. The intriguing result should be further examined on more carefully selected Hot Jupiter samples and put constraints on Hot Jupiter origin channels.

\section{Relations between the $\psi$, $\lambda$, and \lowercase{$i_\star$} distributions}\label{sec:jacobian}

In this section, we aim to understand why the stellar obliquity distribution is predominantly determined by the sky-projected stellar obliquity distribution and why the stellar inclination distribution is less important to the inference using mathematical expressions.

\begin{figure*}[ht!]
    \script{coordinate.py}
    \gridline{
        \fig{figures/coord_psi.pdf}{0.45\textwidth}{\vspace*{-1.8cm}(a) The $\{\psi, \theta\}$ coordinate system. The grey circle corresponds to a constant $\psi$ value and its circumference is proportional to $\sin{\psi}$.}
        \fig{figures/coord_lam.pdf}{0.45\textwidth}{\vspace*{-1.8cm}(b) The $\{\lambda, i_\star\}$ coordinate system. The grey circle corresponds to a constant $i_\star$ value and its circumference is proportional to $\sin{i_\star}$.}
    }
    \vspace*{-1.5cm}
    \caption{Two coordinate systems to describe the stellar spin axis $\bf{n}_{\star}$ and the planet's orbital axis $\bf{n}_{\rm orb}$. Here we define the observer's line of sight as one of the two horizontal axes (conventional $x$-axis in Cartesian), and the orbital axis of the planet as the vertical axis (conventional $z$-axis in Cartesian). We approximate the orbital inclination of the planet to $90\degr$.}
    \label{fig:coord}
\end{figure*}

In Figure~\ref{fig:coord}, we build two coordinate systems to describe the stellar spin axis $\bf{n}_{\star}$ for a given orbital axis $\bf{n}_{\rm orb}$. In both coordinates, we set the observer's line of sight to be one of the two horizontal axes, and the orbital axis of the planet to be the vertical axis. We approximate the orbital inclination of the transiting planet to $90\degr$ here to simplify the problem for a population study. As discussed in Section~\ref{subsec:sims}, the assumption will not compromise the stellar obliquity distribution inference for transiting planets.

The $\{\psi, \theta\}$ coordinate system shown in panel (a) relates to the physical properties of a planetary system. $\psi$ is the angle between the stellar spin axis and the planetary orbital axis, and $\theta$ is the azimuthal angle of the stellar spin axis. For the given $\bf{n}_{\rm orb}$ axis, if $\bf{n}_{\star}$ is a random vector with uniform distribution on a three-dimensional sphere, i.e., $\bf{n}_{\star}$ is isotropically distributed, the probability density function of $\psi$ follows $p_{\psi} \sim \sin{\psi}$ and $p_\theta \sim 1/2\pi$. Since $p_{\cos{\psi}} = p_{\psi} \abs{\dv*{\psi}{\cos{\psi}}}$, $p_{\cos{\psi}}$ is uniformly distributed between $-1$ and $1$ for isotropic stellar obliquity.
The $\{\lambda, i_\star\}$ coordinate system shown in panel (b) relates to observed properties. $\lambda$ is the sky-projected stellar obliquity and $i_\star$ is the stellar inclination. If $\bf{n}_{\star}$ is isotropically distributed, $p_{i_\star} \sim \sin{i_\star}$ and $p_\lambda \sim 1/2\pi$. Again, since $p_{\cos{i_\star}} = p_{i_\star} \abs{\dv*{i_\star}{\cos{i_\star}}}$, $p_{\cos{i_\star}}$ is uniformly distributed between $-1$ and $1$. Because of the observational degeneracy between $i_\star$ and $180\degr - i_\star$, the convention is to have $0 \leq i_\star \leq 90\degr$, and thus $p_{\cos{i_\star}}$ is uniformly distributed between $0$ and $1$ (i.e., $p_{\cos{i_\star}} \sim 1$). This also limits $\theta$ to $[-\pi/2, \pi/2]$. 
In observation, there is also a degeneracy between $-\lambda$ and $+\lambda$. Therefore, we also limit $\lambda$ to $[0,\pi]$.

We could find the mathematical relations between $\{\psi, \theta\}$ and $\{\lambda, i_\star\}$ by pairing the Cartesian components of $\bf{n}_{\star}$ in two coordinate systems:
\begin{align}
    \sin{\psi}\cos{\theta} = \cos{i_\star}& \label{eq:coord1}\\
    \sin{\psi}\sin{\theta} = \sin{\lambda}\sin{i_\star}& \label{eq:coord2}\\
    \cos{\psi} = \cos{\lambda}\sin{i_\star} \label{eq:coord3}&.
\end{align}
Thus, $i_\star = \cos[-1](\sin{\psi}\cos{\theta})$ and $\lambda = \tan[-1](\tan{\psi}\sin{\theta})$. 

First, we derive the $\lambda$ distribution for a given $\cos{\psi}$ distribution. We could find the distribution of $\cos{\lambda}$ using the Jacobian transformation from $\cos{\psi}$ and $\cos{\theta}$. Since $\psi$ and $\theta$ are independent variables, we could marginalize over $\theta$ to find the relation between the probability density functions between $\lambda$ and $\psi$.
The Jacobian transformation follows
\begin{equation}
    p_{\cos{\lambda}} = \int \abs{\pdv{\cos{\psi}}{\cos{\lambda}}} p_{\cos{\psi}} p_{\cos{\theta}} d\cos{\theta}.
\end{equation}
Replacing $\sin{i_\star}$ in Equation~(\ref{eq:coord3}) with Equation~(\ref{eq:coord1}), we find $\cos{\lambda} = \cos{\psi}/\sqrt{1-(1-\cos^2{\psi})\cos^2{\theta}}$. Reorganize the equation, we get $\cos^2{\psi} = \frac{\cos^2{\lambda}\cos^2{\theta}-\cos^2{\lambda}}{\cos^2{\lambda}\cos^2{\theta}-1}$.
The derivative gives $\abs{\pdv{\cos{\psi}}{\cos{\lambda}}} = \frac{(1-\cos^2{\theta})^{1/2}}{(1-\cos^2{\theta}\cos^2{\lambda})^{3/2}}$. If $\theta$ is uniformly distributed between $-\pi/2$ and $\pi/2$, $p_{\cos{\theta}} = p_{\theta} \abs{\dv*{\theta}{\cos{\theta}}} = 1/\pi/(1-\cos^2{\theta)^{1/2}}$. Putting all the parts together, we get
\begin{equation}\label{eqn:jac_lam}
    p_{\cos{\lambda}} = \frac{2}{\pi} \int_{0}^{1} (1-\cos^2{\theta}\cos^2{\lambda})^{-3/2} p_{\cos{\psi}} d\cos{\theta}.
\end{equation}
If $\cos{\psi}$ is uniformly distribution, i.e., $p_{\cos{\psi}} = 1/2$, Equation~(\ref{eqn:jac_lam}) becomes $p_{\cos{\lambda}} = 1/\pi/\sqrt{1-\cos^2{\lambda}}$, which is equivalent to $p_\lambda = 1/\pi$. The $\lambda$ is uniformly distributed if an isotropic $\psi$. If $\cos{\psi}$ follows other distributions, $p_\lambda$ could be found by evaluating Equation~(\ref{eqn:jac_lam}) numerically.

Next, we derive the $i_\star$ distribution for a given $\cos{\psi}$ distribution. Similarly, we first find the Jacobian transformation of $i_\star$ from $\psi$ and $\theta$ and then marginalize over $\theta$. It is easier to work on $\cos{i_\star}$ than $i_\star$:
\begin{equation}
    p_{\cos{i_\star}} = \int \abs{\pdv{\sin{\psi}}{\cos{i_\star}}} p_{\sin{\psi}} p_{\cos{\theta}} d\cos{\theta}.
\end{equation}
From Equation~(\ref{eq:coord1}), we have $\sin{\psi} = \cos{i_\star}/\cos{\theta}$ and $\abs{\pdv{\sin{\psi}}{\cos{i_\star}}} = 1/\cos{\theta}$. Again, we assume $\theta$ is uniformly distributed, and $p_{\cos{\theta}} = 1/\pi/(1-\cos^2{\theta)^{1/2}}$. Lastly, we transform the $p_{\sin{\psi}}$ to $p_{\cos{\psi}}$, where $p_{\sin{\psi}} = p_{\cos{\psi}}\sqrt{1-\cos^2{\psi}}/\cos{\psi}$. Combining all the pieces together, we get
\begin{equation}\label{eqn:jac_istar}
    p_{\cos{i_\star}} = \frac{4}{\pi} \int_{\cos{i_\star}}^{1} \frac{\cos{i_\star}/\cos{\theta}}{\sqrt{\cos^2{\theta}-\cos^2{i_\star}}} \frac{1}{\sqrt{1-\cos^2{\theta}}} p_{\cos{\psi}} d\cos{\theta}.
\end{equation}
If $\cos{\psi}$ is uniformly distribution, i.e., $p_{\cos{\psi}} = 1/2$, the integral gives $1$, which suggests the $\cos{i_\star}$ probability density function is uniform.

From Equation~(\ref{eqn:jac_lam}) and (\ref{eqn:jac_istar}), we can now derive the $\lambda$ and $i_\star$ distributions for any given $\psi$ distributions, assuming the azimuthal angle of the stellar spin axis $\theta$ is random. In Figure~\ref{fig:transform}, we find and examine numerical solutions of the $\lambda$ and $i_\star$ distributions for four different $\cos{\psi}$ distributions. The first row in Figure~\ref{fig:transform} presents an isotropic $\psi$ where $\cos{\psi} \sim \mathcal{U}(-1,1)$, and the second to fourth rows in Figure~\ref{fig:transform} present Normal distributions of $\cos{\psi}$ following $\mathcal{N}(0,0.2)$, $\mathcal{N}(-0.4,0.2)$, and $\mathcal{N}(0.4,0.2)$, respectively. In each row, we solve the $\lambda$ and $i_\star$ distributions numerically as shown in blue curves, and samplings of $\lambda$ and $i_\star$ from $\cos{\psi}$ and $\theta$ distributions as shown in grey histograms. For a uniform $\cos{\psi}$ distribution, the $\lambda$ distribution is uniform, and the $i_\star$ distribution is isotropic and proportional to $\sin{i_\star}$, as expected.

Interestingly, for a given stellar obliquity distribution, the $\lambda$ distribution is sensitive to and closely related to the underlying $\psi$ distribution, as shown in Figure~\ref{fig:transform}. For different stellar obliquity distributions, the $\lambda$ distributions are distinguishable from each other, and thus, when we infer the $\psi$ distribution from the $\lambda$ distribution, the solution will not degenerate.
On the other hand, the $i_\star$ distributions less depend on the underlying $\psi$ distribution. Compared to an isotropic $i_\star$ distribution, the $i_\star$ distributions for different $\psi$ distributions differ at the low values (i.e., $i_\star < \pi/4$), which places a challenge to observation. More importantly, since the $i_\star$ distributions are less sensitive to the underlying $\psi$ distributions when we infer the $\psi$ distributions from the $i_\star$ distribution, the solution could be very degenerate.

Due to the lack of dependency on the $i_\star$ distribution, the $\psi$ distribution could be inferred purely from the $\lambda$ distribution without losing much information. Although we could find a relation between $\psi$, $\lambda$, and $\theta$, the $\psi$ distribution cannot be inferred from the $\lambda$ and $\theta$ distributions since $\lambda$ and $\theta$ are not independent variables. We still need to infer the $\psi$ distribution from the $\lambda$ and $i_\star$ distributions.
In the \nth{4} column in Figure~\ref{fig:transform}, we find the $\psi$ distribution from the $\lambda$ distribution, simply assuming an isotropic $i_\star$ distribution. The method could correctly recover the peaks and widths of the underlying $\psi$ distributions (presented as blue dashed lines) as shown in grey histograms. The method over-predicts $\cos{\psi}$ at $\pm 1$ because of the over-prediction of $i_\star$ near $90\degr$. However, this minor deviation from the true distribution has little impact on the population inference because of hyperparameters' limited degree of freedom. Comparing the blue curves with the orange curves in Figure~\ref{fig:simulation}, the overestimation of $\cos{\psi}$ near $\pm1$ under the assumption of an isotropic $i_\star$ has little impact on the inference since the distribution is mainly constrained by the peak and width of the stellar obliquity distribution, which are both correctly predicted by the isotropic $i_\star$ model.

\begin{figure*}[ht!]
    \script{transform.py}
    \includegraphics[width=\linewidth]{figures/transform.pdf}
    \caption{Simulated $\cos{\psi}$ distributions (\nth{1} column) and their corresponding sky-projected stellar obliquity $\lambda$ (\nth{2} column) and stellar inclination $i_\star$ (\nth{3} column) distributions. The inferred $\cos{\psi}$ distributions assuming isotropic stellar inclinations are shown in the \nth{4} column. The random samplings of $\lambda$ and $i_\star$ out of the $\cos{\psi}$ distributions are shown as grey histograms, and the numerical solutions are shown as blue curves.}
    \label{fig:transform}
\end{figure*}

\section{Summary \& Discussion}

In this work, we demonstrated that the stellar obliquity distribution inference is predominantly determined by the sky-projected stellar obliquities. Without stellar inclination measurements, the stellar obliquity distribution could still be correctly inferred.
We introduce a flexible, hierarchical Bayesian framework to infer the stellar obliquity distribution of transiting-planet systems. Stellar inclination measurement is optional, and if it is not provided, we simply assume an isotropic stellar inclination distribution. The hierarchical Bayesian model is an open-source code that can be found on \href{https://github.com/jiayindong/obliquity}{GitHub\,\faGithub\,(https://github.com/jiayindong/obliquity)}. Users may apply the framework to a certain sample of targets and customize the model for different stellar obliquity distributions and priors.
When jointly modeling the stellar obliquity distribution from one sample with stellar inclination measurements and one without, it is important to check if the $i_\star$ sample is representative of the full sample. If not, since the $i_\star$ sample will constrain stellar obliquities more tightly, the overall distribution could be weighted towards these systems, bias the interpretation.
Lastly, we applied the framework to all exoplanetary systems with sky-projected stellar obliquity measurements and found that $67\pm9$\% of the systems have stellar obliquity less than $40\degr$, and $33\pm9$\% of the systems have nearly isotropically distributed obliquities between $40\degr$ and $140\degr$.
The distribution could have important implications for the formation and evolution of close-in planetary systems and be worthwhile for further investigation.

\vspace*{5mm}

\software{$\mathtt{ArviZ}$ \citep{arviz_2019}, $\mathtt{Jupyter}$ \citep{Jupyter}, $\mathtt{Matplotlib}$ \citep{Matplotlib07, Matplotlib16}, $\mathtt{NumPy}$ \citep{NumPy11, NumPy20}, $\mathtt{pandas}$ \citep{mckinney-proc-scipy-2010, reback2020pandas}, $\mathtt{PyMC}$ \citep{pymc}, $\mathtt{SciPy}$ \citep{2020SciPy-NMeth}}

\bibliography{bib}

\end{document}
