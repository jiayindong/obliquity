% Define document class
\documentclass[twocolumn]{aastex631}
\usepackage{showyourwork}

% Begin!
\begin{document}

% Title
\title{True Stellar Obliquity Distribution of Exoplanets}

\correspondingauthor{Jiayin Dong}
\email{jdong@flatironinstitute.org}

\newcommand{\FlatironCCA}{Center for Computational Astrophysics, Flatiron Institute, 162 Fifth Avenue, New York, NY 10010, USA}

\author[0000-0002-3610-6953]{Jiayin Dong}
\altaffiliation{Flatiron Research Fellow}
\affiliation{\FlatironCCA}

\author[0000-0002-9328-5652]{Dan Foreman-Mackey}
\affiliation{\FlatironCCA}

% Abstract with filler text
\begin{abstract}
We build a hierarchical Bayesian framework to constrain exoplanets' true stellar obliquity ($\psi$) distribution. Using the sky-projected stellar obliquities ($\lambda$)  of a sample of 162 exoplanets, we infer their $\cos{\psi}$ distribution using a four-parameter Beta mixture model under the assumption of a uniform cosine of the stellar inclination ($\cos{i_\star}$) bounded between $-1$ and $1$. We find a significant pile-up of $\cos{\psi}$ near 1, suggesting that the majority of exoplanetary systems are well aligned. For misaligned systems, we find $\cos{\psi}$ is uniformly distributed between $-0.75$ to $0.5$ and have a slight but insignificant preference to $\cos{\psi} = -0.25$, corresponding to $\psi = 104.5\degr$. A pile-up at $\cos{\psi} = 0$ is not found.
\end{abstract}

% Main body with filler text
\section{Introduction}
\label{sec:intro}

\bibliography{bib}

\end{document}
